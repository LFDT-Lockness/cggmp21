\documentclass[11pt]{article}
%\usepackage[utf8]{inputenc}
\usepackage[noend]{algpseudocode}
\usepackage{algorithm}
\usepackage{fullpage}
\usepackage{amsmath,amsfonts,amssymb}
\usepackage[dvipsnames]{xcolor}
\usepackage{graphicx,etoolbox}
\usepackage{hyperref,url}

\newtoggle{darkMode}
% \toggletrue{darkMode}
\togglefalse{darkMode}

\definecolor{DfnsPurple}{RGB}{19,7,34}
\definecolor{DfnsWhite}{RGB}{254,249,239}

\iftoggle{darkMode}{
    \pagecolor{DfnsPurple}
    \color{DfnsWhite}
}{
    \pagecolor{DfnsWhite}
    \color{DfnsPurple}
}

\def\crt{{\sf crt}}
\def\tmp{{\sf tmp}}
\def\res{{\sf res}}
\newcommand{\ignore}[1]{}
\newcommand{\jnote}[1]{{\textcolor{blue}{Jon's note: #1}}}
\newcommand{\dnote}[1]{{\textcolor{Orange}{Denis: #1}}}
\newcommand{\nnote}[1]{{\textcolor{purple}{Mikito: #1}}}

%\newgeometry{vmargin={8mm,16mm}, hmargin={12mm,12mm}}

\def\state{{\sf state}}
\def\eps{\varepsilon}
%\def\order{{\tt curve}\verb+_+{\tt order}}
\newcommand{\threshold}{\ensuremath{\mathtt{min\_signers}}}
\newcommand{\fvss}{\ensuremath{\mathtt{vss}}}
\newcommand{\enc}{\ensuremath{\mathtt{enc}}}
\newcommand{\enci}[1]{\ensuremath{\mathtt{enc}_{N_{#1}}}}
\newcommand{\dec}{\ensuremath{\mathtt{dec}}}
\newcommand{\deci}{\ensuremath{\mathtt{dec}}_{\sk_i}}
\newcommand{\encCrt}
{{\ensuremath{\mathtt{enc}}^\crt_{\sk_i}}}
\newcommand{\encCrti}[1]{{\ensuremath{\mathtt{enc}}^\crt_{\sk_{#1}}}}
\newcommand{\sid}{\ensuremath{\mathtt{sid}}}
\newcommand{\proof}[1]{\ensuremath{\Pi^{\mathtt{#1}}}}
\newcommand{\commit}[1]{\ensuremath{\mathtt{Commit}_{\mathtt{#1}}}}
\newcommand{\challenge}[1]{\ensuremath{\mathtt{Challenge}_{\mathtt{#1}}}}
\newcommand{\challengeni}[1]{\ensuremath{\mathtt{ChallengeNI}_{\mathtt{#1}}}}
\newcommand{\prove}[1]{\ensuremath{\mathtt{Prove}_{\mathtt{#1}}}}
\newcommand{\proveni}[1]{\ensuremath{\mathtt{ProveNI}_{\mathtt{#1}}}}
\newcommand{\verify}[1]{\ensuremath{\mathtt{Verify}_{\mathtt{#1}}}}
\newcommand{\verifyni}[1]{\ensuremath{\mathtt{VerifyNI}_{\mathtt{#1}}}}
\newcommand{\ou}{\text{\;or\;}} % \or is prohibited in latex for some reason
\newcommand{\Encode}[1]{\ensuremath{\mathtt{Encode}_{\mathtt{#1}}}}

\newcommand{\changedt}[1]{\colorbox{Salmon}{#1}}
\newcommand{\changed}[1]{\colorbox{Salmon}{$\displaystyle #1$}}
\newcommand{\note}[1]{\colorbox{Goldenrod}{$\displaystyle #1$}}

\newcommand{\affineX}[1]{#1|_{\textit{x}}}

\newcommand{\sk}{\textsf{sk}}
\newcommand{\pk}{\textsf{pk}}
\newcommand{\rid}{\textsf{rid}}

\newcommand{\Fq}{\mathbb{F}_q}
\newcommand{\E}{\mathbb{E}}
\newcommand{\Z}{\mathbb{Z}}

\newcommand{\?}[1]{\stackrel{?}{#1}}

\newcommand{\todo}[1]{\textbf{\color{Red}{TBD: #1}}}

\renewcommand{\labelitemi}{--}

\title{{\bf CGGMP Specification}}
\author{
    \includegraphics[width=50pt]{images/DFNS-logo.png}
}
\date{}

\begin{document}

\maketitle

\section{Introduction}
We provide a specification for our implementation of the CGGMP threshold signing protocol~\cite{cggmp21}. 

\section{Notation and Preliminaries}
$\E$ denotes an elliptic-curve group of prime order\footnote{The curve order is denoted by {\tt curve}\_{\tt order} in the code. Note that $q$ is also used for the verifier's challenge space in~\cite{cggmp21}, but we use $Q$ for that instead.}~$q$ with generator $G$. %(Note we write $\E$ additively, whereas multiplicative notation is used in~\cite{cggmp21}.) 
If $P \in \E$ is a point on the curve, then $\affineX{P}$ denotes the $x$-coordinate of~$P$. We let $\Z_n=[n]=\{0, \dots, n-1\}$, and let $\Z_n^*$ be the subset of elements of~$\Z_n$ co-prime to~$n$, i.e., $\Z^*_n=\{i \mid i \in \Z_n \land \gcd(i, n) = 1\}$. 
For integers $a, b, \ell$, we set 
$[a; b)=\{a, \ldots, b-1\}$
and
$\pm \ell=\{-\ell, \dots, 0, \dots, \ell\}$. 
We write $x \gets X$ to denote sampling a uniform element $x$ from a set~$X$.
%, and $x \stackrel{s}{\gets} X$ to denote the process of deterministically mapping a seed $s$ to an element of $X$ that is computationally indistinguishable from uniform (assuming $s$ is uniform and unknown).

\def\isprime{{\sf IsPrime}}

\subsection{Safe-Prime Generation}
\label{sec:safe-prime}
A safe prime $p$ has the form $p=2p'+1$ where $p'$ is also prime. 
Assuming a primality test $\isprime$, 
a trivial way to generate a (random) safe prime is 
to repeatedly sample $p'$ in the appropriate range, test $p'$ for primality, and then (if $p'$ is prime) test $2p'+1$ for primality.
(We ignore here the possibility of error in $\isprime$.)

\ignore{
\begin{algorithm}
\caption{Generating a (random) safe prime}\label{alg:primegen-trivial}
\begin{algorithmic}[1]
\While{(1)}
\State Choose (random) $p'$ in the appropriate range
\State if $(!\isprime(p'))$ {\sf continue}
\State if $(\isprime(2p'+1))$ \Return $p=2p'+1$
\EndWhile
\end{algorithmic}
\end{algorithm}
}

Primality testing is expensive, and the trivial algorithm is wasteful in the sense that it tests $p'$ for primality even when it is clear that $p=2p'+1$ will not be prime (e.g., if $p'=1 \bmod 3$).
We can avoid this by using simple sieving, as in Algorithm~\ref{alg:primegen} (following~\cite{safe-primes}).

\begin{algorithm}
\caption{Generating a (random) safe prime}\label{alg:primegen}
\begin{algorithmic}[1]
\State Let $B$ be a set containing the first $n$ odd primes \;\;\; // $n$ is a parameter
\While{(1)}
\State Choose (random) $p'$ in the appropriate range
\State If $p' \bmod q \in \{0, (q-1)/2\}$ for some $q \in B$ {\sf continue}
\State If $(!\isprime(p'))$ {\sf continue}
\State If $(\isprime(2p'+1))$ \Return $p=2p'+1$
\EndWhile
\end{algorithmic}
\end{algorithm}

The number of primes $n$ to use for sieving is a parameter that can be heuristically optimized. Note that as $n$ increases, the marginal benefit of sieving decreases while the cost of sieving increases.

\subsection{Using the Chinese Remainder Theorem}
\label{sec:CRT}
Arithmetic modulo $N$ can be optimized when the (partial) factorization of $N$ is known.
Say $N=N_1\cdot N_2$ where $N_1, N_2 > 1$ and $\gcd(N_1, N_2)=1$. (Note that $N_1, N_2$ need not be prime.)
Then a computation modulo $N$ can be optimized by (1)~separately carrying out the computation modulo~$N_1$ and~$N_2$, and then (2)~combining the results.
We illustrate the for the particular case of exponentiation (i.e., computing
$s^x \bmod N$), but the same idea can be applied for multiplication, multiexponentiation, etc.

\medskip\noindent{\bf Exponentiation modulo $N_1, N_2$.}
Computing $s^x \bmod N_1$ and $s^x \bmod N_2$ will, in general, be faster than computing $s^x \bmod N$ directly because (1)~$N_1, N_2$ are shorter than~$N$, and (2)~assuming the factorizations of $N_1, N_2$ are known, we can reduce the exponent~$x$ modulo $\phi(N_1)$ (resp.,~$\phi(N_2)$) before performing the respective exponentiations.
Namely, we can use the fact that, e.g., 
\[s^x \bmod N_1 = s^{x \bmod \phi(N_1)} \bmod N_1\]
(assuming $\gcd(s, N_1)=1$.)

\medskip\noindent{\bf Combining the results.}
Let $\beta=N_1^{-1} \bmod N_2$. %(Note this can be computed as part of a preprocessing step.) 
If we have computed $r_1=s^x \bmod N_1$ and $r_2=s^x \bmod N_2$, we can compute $s^x \bmod N$ as
\[r_1+\left((r_2-r_1) \cdot \beta \bmod N_2\right) \cdot N_1 .\]
(Note that no reduction modulo $N$ is needed, and the result will already be in the correct range.)
To see that this gives the correct answer, note that 
\begin{eqnarray*}
& r_1+\left((r_2-r_1) \cdot \beta \bmod N_2\right) \cdot N_1 =  r_1 \bmod N_1 &\\
& r_1+\left((r_2-r_1) \cdot \beta \bmod N_2\right) \cdot N_1 = r_1+\left((r_2-r_1) \cdot N_1^{-1} \right) \cdot N_1 = r_2 \bmod N_2. &
\end{eqnarray*}
%then apply the Chinese remainder theorem.


\medskip
In our context, the case of interest is when the modulus is $N^2=p^2 q^2$ and the factors $p, q$ are known.
Algorithm~\ref{alg:CRT} shows a complete algorithm for exponentiation modulo~$N^2$ in that case.

\begin{algorithm}
\caption{Computing $s^x \bmod N^2$, where $N^2=p^2q^2$ with $p, q$ distinct primes and $\gcd(s,N^2)=1$}\label{alg:CRT}
\begin{algorithmic}[1]
\State $\beta:=p^{-2} \bmod q^2$ \;\;\; //  this can be computed in a preprocessing step
\State $\phi_1 = p \cdot (p-1), \; \phi_2 := q \cdot (q-1)$
\State $x_1:= x \bmod \phi_1, \; x_2:=x \bmod \phi_2$
\State $s_1 := s \bmod p^2, \; s_2 := s \bmod q^2$
\State $r_1:= s_1^{x_1} \bmod p^2, \; 
r_2 := s_2^{x_2} \bmod q^2$
\State $\res := r_1+((r_2-r_1)\cdot \beta \bmod q^2) \cdot p^2$
\State \Return \res
\end{algorithmic}
\end{algorithm}


\subsection{Paillier Encryption Scheme}
We include a description of the algorithms that constitute the Paillier encryption scheme. 
%Note, however, that we call an external library for these algorithms, and do not implement them ourselves.
\begin{enumerate}
    \item $\textsf{keygen}$ generates a private key $\sk$ consisting of two safe primes, with the public key being their product~$N$.
    
    
    \item 
    $\enc_N(M; r)$ 
    encrypts a message $M \in \{-(N-1)/2, \ldots,  (N-1)/2\}$ 
    using randomness $r \in \Z_N^*$ 
    and Paillier public key~$N$. 
    This produces a ciphertext $C \in \Z^*_{N^2}$.
    Encryption checks that $\gcd(r, N)=1$ (and raises an error if not), and then computes
    
    $$\enc_N(M; r) := (1 + M \cdot N) \cdot r^N \bmod N^2$$ 

        We also provide a function $\enc_N(M)$ that samples uniform $r \in \Z_N^*$ and returns~$\enc_N(M; r)$.

    Functions $\enc^{{\sf crt}}_{\sk}(M;r)$ and $\enc^{{\sf crt}}_{\sk}(M)$ are analogous but achieve better performance by using the technique from Section~\ref{sec:CRT} when the factorization of $N$
    is known.

    \item $\dec_{\sk}(C)$ decrypts a ciphertext $C \in \Z^*_{N^2}$ to a plaintext $M \in \{-(N-1)/2, \ldots,  (N-1)/2\}$.

    \item $C_1 \oplus C_2$ denotes homomorphic addition of ciphertexts $C_1, C_2 \in \Z^*_{N^2}$ encrypted under the same Paillier public key~$N$. It is computed as $C_1 \oplus C_2 = C_1 \cdot C_2 \bmod N^2$. Note that 
$\dec(C_1 \oplus C_2) = [\dec(C_1) + \dec(C_2) \bmod N]$. 

    

    \item $k \odot C$ denotes homomorphic multiplication of a ciphertext $C \in \Z^*_{N^2}$ by $k \in \Z$. It is computed as $k \odot C = C^k \bmod N^2$. Note that $\dec(k \odot C) = [k \cdot \dec(C) \bmod N]$.

    
\end{enumerate}

\subsection{Speeding up Fixed-Based Multiexponentiation Using Preprocessing}
\label{sec:multiexp}
Execution of the protocol involves many computations of the form $s^x t^y \bmod N$, where $s, t, N$ are fixed (and known in advance) but the exponents $x, y$ vary. 
For the purposes of this section we view this as a multiexponentiation with respect to the bases $s, t$ in a generic group, and so ignore~$N$.
Efficiency of these multiexponentiations can be improved by using \emph{one-time preprocessing} to generate a small amount of state that is used to speed up subsequent computations.

Say $-\ell_x < x < \ell_x$ and $-\ell_y < y < \ell_y$, where typically $\ell_x, \ell_y$ are powers of~2.
In describing the algorithm, we assume $x, y$ are nonnegative; 
we can handle negative exponents by also
pre-computing $s^{-\ell_x}$ and $t^{-\ell_y}$ and then, e.g., when $x$ is negative write $s^xt^y = \left(s^{-\ell_x}\right) \cdot \left(s^{x+\ell_x}t^y\right)$ with \mbox{$x+\ell_x> 0$}.
The algorithm is parameterized by a value $B$ which is also typically a power of~2 (in practice, taking $B\in\{2^4,2^8\}$ is a good choice); it stores $T_B \approx (\log \ell_x + \log \ell_y)/\lg B$ group elements and requires 
$\approx B+T_B$ group operations to compute a multiexponentiation.
See Algorithm~\ref{alg:multiexp-preprocess2a}.

\begin{algorithm}
\caption{Computing $s^xt^y$; parameterized by $B\geq 2$; let $k'_x=\lceil |x|/\lg B\rceil$, $k'_y=\lceil |y|/\lg B\rceil$}\label{alg:multiexp-preprocess2a}
\begin{algorithmic}[1]
%\Require $x,y \geq 0$
\State in preprocessing step, compute $s_{i} := s^{B^i}$ for $i\in \{0, \ldots, k_x'-1\}$ 
\State in a preprocessing step, compute $t_i := t^{B^i}$ for $i\in \{0, \ldots, k_y'-1\}$
%\If{$x<0$}
%\State $\neg:=1$, $x:=-x$
%\Else \State $\neg:=0$
%\EndIf
\State let $x_{k_x'-1} \cdots x_0$ and $y_{k'_y-1}\cdots y_0$ be the base-$B$ representations of~$x$ and~$y$, respectively
\State $\res := 1$, $\tmp:=1$
\For{$b=B-1, \ldots, 1$}
  \For{all $i$ such that $y_i=b$}
    \State $\tmp:= \tmp \cdot t_i$
   \EndFor
   \For{all $i$ such that $x_i=b$}
     %\If{$\neg=0$}
    \State $\tmp:= \tmp \cdot s_i$
    %\Else \State $\tmp:=\tmp \cdot s_i^{-1}$
    %\EndIf
   \EndFor
   \State $\res:=\res\cdot \tmp$
\EndFor
\State \Return \res
\end{algorithmic}
\end{algorithm}

%\jnote{Use signed-digit representation?Storage $2(k_x+k_y)/\lg B$, time $B/2+(k_x+k_y)/\lg B$}

We refer to $(\{s_i\}_{i=0}^{k'_x}, \{t_i\}_{i=0}^{k'_y})$ as a \emph{table}~$T_i$.
Precomputation of a table is only done once, so the efficiency of doing so is not critical; nevertheless, 
for completeness, we describe an algorithm for computing the $\{s_i\}_{i \in \{0, \ldots, k'_x-1\}}$. (The same algorithm can be used for computing the $\{t_i\}$ as well.)

\begin{algorithm}
\caption{Computing $\{s_i\}_{i=0}^k$, where $s_i = s^{B^i}$}
\begin{algorithmic}[1]
\State $s_0:=s$
\For{$i=1, \ldots, k$}
    \State $s_i:=s_{i-1}^B$
\EndFor
\end{algorithmic}
\end{algorithm}

\noindent
Assuming $B$ is a power of~2, the exponentiation in line~3 requires $\lg B$ squarings; the algorithm thus uses only $k\lg B$ squarings overall.

\subsection{Security Parameters}
The protocol relies on several user-defined parameters that determine its security. Note  these do not include the curve order~$q$, which is fixed by the underlying signature scheme  rather than by the threshold protocol itself.
We let $\lambda$ denote the bit length of the curve order (so $2^\lambda \leq q < 2^{\lambda+1}$) and assume $\lambda\geq 256$ (which is the case for the signature schemes we support).

The security parameters of the protocol are denoted collectively by $L = (\kappa, \varepsilon, \ell, \ell', m, Q)$. 
These parameters are used in the following ways:
\begin{itemize}
    \item $\kappa$ determines the length of the primes used for Paillier private keys. Specifically, the primes are chosen to be of length~$4\kappa$ and so the Paillier modulus has length~$8\kappa$.

    \item $\ell, \ell'$ correspond to bounds on the ranges of certain plaintexts that are encrypted, while %Parameters should be set so that $2^\ell \leq 2^{8\kappa-1}$.
    $\eps$ is a slack parameter. (Honest parties choose plaintexts in a range determined by $\ell$ or $\ell'$; the zero-knowledge proofs, however, only prove that a party chose plaintexts in a range determined by $\ell+\eps$ or $\ell'+\eps$.)

    \item $m$ denotes the number of iterations of some underlying zero-knowledge protocol to run; the soundness error will be~$2^{-m}$.

    \item $Q$ determines the challenge space for some of the zero-knowledge proofs.
\end{itemize}
For correctness, we require
$\ell \geq \lambda$,
    $\epsilon \geq 8 + \log Q$, and
    $\ell' \leq 8\kappa$.
We also recommend $Q=2^m$ since it can only hurt efficiency (while not improving security) otherwise. %(We could even hardcode this, and allow the user to set~$m$ but not~$Q$.)

\subsubsection{Security Guidelines}
Let $s\leq 256$ be a statistical security parameter, so the goal is to achieve roughly $2^{-s}$ 
``privacy loss'' in one execution of the protocol. %(Roughly speaking, this relates to the statistical difference between real and simulated executions.) 
Parameters can be set using the following guidelines: 
% CGGMP paper uses m=80, \ell=\lambda, \ell'=5\lambda, \epsilon=2\lambda
\begin{itemize}
    \item $\kappa$ should be set based on current estimates regarding hardness of factoring. 
    Setting $\kappa=384$ (so moduli are 3072-bits long) matches NIST recommendations for achieving 128-bit computational security, which is consistent with the security obtained by using~$\lambda=256$.
    
    \item $\ell$ needs to be set such that $2^{\ell+1}\geq q$; setting $\ell \geq \lambda$ ensures this.
Some of the zero-knowledge proofs have privacy loss and soundness error at least~$2^{-\ell}$, but since $\ell = \lambda \geq s$ that is fine.
    
    \item $Q$ and $m$ determine the soundness error of several of the zero-knowledge proofs, with some of the proofs having soundness error at least~$1/Q$ and others having soundness error at least~$2^{-m}$. It thus makes sense to set $Q=2^m$ (as recommended above).    
    Setting $Q=2^{128}$ (and $m=128$) suffices for 128-bit security.  {\bf Note:} while it is possible to increase $Q$ without any significant direct impact on efficiency, increasing $Q$ requires increasing $\epsilon, \ell'$, which does impact efficiency. 
    
    \item $\epsilon$ affects both the completeness error and the privacy loss of several of the zero-knowledge proofs. 
    %For correctness, we minimally require $\epsilon \geq 2+\log Q$. Moreover, 
    Since 
    some proofs have privacy loss at least~$4Q/2^\epsilon$, this requires $\epsilon \geq 2+s+\log Q$.

    \item $\ell'$ needs to be set large enough so that adding noise from $\pm 2^{\ell'}$ statistically hides a $(2\lambda+\epsilon)$-bit value. This requires $\ell' \geq 2\lambda + \epsilon + s$.
\end{itemize}
If the above guidelines are used, 
the interaction between one honest party and one malicious party during an execution of the signing protocol has privacy loss upper-bounded by $8 \cdot 2^{-s}$ (this accounts for all the zero-knowledge proofs as well as the noise used for statistically hiding different values). 
If we assume $t-1$ malicious parties and $n-t+1$ honest parties, the overall privacy loss in an execution of the protocol is at most $8\cdot (n-t+1)\cdot (t-1) \cdot 2^{-s}$.
%If we assume any particular key is used at most $2^{20}>10^6$ times over its lifetime, then the aggregate privacy loss for any particular key is at most $2^{23-s} \cdot (t-1) (n-t+1)$.
%Bounding the aggregate privacy by $2^{-64}$ is reasonable.

\subsection{Unambiguous encoding}
We require unambiguous encode function 

$$\Encode{tag} : X_1 \times X_2 \times \dots \times X_n \to \{0, 1\}^*$$

which takes arguments $x_1 \in X_1, \dots, x_n \in X_n$ and produces their unambiguous bytes encoding.
$X_i$ can be any domain that has bytes representation. E.g. if we encode an integer, then $X_i$ is set
of all integers, or if we encode a UTF-8 string, then $X_i$ is set of all valid UTF-8 strings.

Encoding function should satisfy these properties:
\begin{itemize}
\item {\bf Unambiguousy.} For fixed $x_1 \in X_1, \dots, x_n \in X_n$, $\Encode{tag}(x_1, \dots, x_n)$
    outputs a bytestring unique for input arguments, i.e.:

    \begin{equation*}
    \begin{split}
    & \forall x_1 \in X_1, \dots, x_n \in X_n \\
    & \forall x'_1 \in X_1, \dots, x'_n \in X_n: (x_1 \ne x'_1 \lor \dots \lor x_n \ne x'_n) \\
    & \Encode{tag}(x_1, \dots, x_n) \ne \Encode{tag}(x'_1, \dots, x'_n)
    \end{split}
    \end{equation*}
\item {\bf Tag-dependence.} Tag should be unambiguously encoded into the output, so
    $\Encode{tag}(\dots) \ne \Encode{tag'}(\dots)$ for any $\text{tag} \ne \text{tag}'$. 
\item {\bf Platform-independence.} Encoding should be the same regardless of platform or machine on which it's
    executed.
\end{itemize}

\section{Zero-Knowledge Proofs}
In this section we describe the various zero-knowledge proofs that are used as sub-routines in the protocol. In each case we first describe an interactive version of the proof; we then describe how we implement a non-interactive version of the proof using the Fiat-Shamir transform.

\subsection{Non-Interactive Challenge Derivation}
In order to implement Fiat-Shamir transform correctly, we need to deterministically and securely derive a challenge
from the inputs. To do so, we define a function:

$$\challengeni{tag} : X_1 \times \dots \times X_n \to Y_1 \times \dots \times Y_k$$

which takes proof inputs and outputs uniformelly distributed challenges (number of challenges to be
sampled depends on the proof).

To sample the challenges, we first construct the cryptographic pseudo-random generator $G$:

\begin{align*}
G = & H(\Encode{tag}(0, x_1, \dots, x_n)) \\
    & \| \; H(\Encode{tag}(1, x_1, \dots, x_n)) \\
    & \| \; H(\Encode{tag}(2, x_1, \dots, x_n)) \\
    & \| \; \dots
\end{align*}

And then we use pseudo-random stream $G$ to deterministically sample $y_1 \gets Y_1, \dots, y_k \gets Y_k$.

{\bf Note:} sometimes we write $\challengeni{tag}^L(\dots)$ which is equivalent to $\challengeni{tag}(L, \dots)$.

\subsection{$\proof{enc}$: Paillier Encryption in Range}

We assume the prover and verifier agree on shared state $\state$, auxiliary data $R_j = (N_j, s_j, t_j)$ with $s_j, t_j \in \Z^*_{N_j}$, and a security level~$L$. 
The prover and verifier have common input $(N_i, K)$, and the prover additionally has 
secret input $(k, \rho)$ such that $k \in \pm 2^\ell$ and $K = \enc_{N_0}(k; \rho)$.
%\hat N => N_j
%N_0 => N_i
In all the cases where this proof is used in the protocol, the prover knows the factorization of~$N_i$ (and thus knows~$\sk_i$) and the verifier knows the factorization of~$N_j$ (and thus knows $\sk_j$).
  %  \item Proof guarantees: $k \in \pm 2^{\ell + \varepsilon}$ 
 %   \item Implicit constraints: $2^{\ell + \varepsilon} \le N_0/2$ 
%\end{itemize}

\subsubsection{Interactive Version of the Proof}

\begin{enumerate}
    \item In the first round of the protocol, the prover does the following:
    %$\commit{enc}^{L,R}((N_0, K); (k, \rho)) \to ((S, A, C); (\alpha, \mu, r, \gamma))$:
    \begin{itemize}
        \item The prover  samples the following values:
        \[\alpha \gets \pm 2^{\ell + \varepsilon}, \;\;\;
        \mu \gets \pm (2^\ell \cdot N_j), \;\;\;
        r \gets \Z_{N_i}^*, \;\;\;
        \gamma \gets \pm (2^{\ell + \varepsilon} \cdot N_j).\]

        \item The prover then computes:
\begin{itemize}
        \item $S = s_j^k t_j^\mu \bmod N_j$ 
            \item $A = \enc_{N_i}(\alpha; r)$  (this is computed as $\enc^\crt_{\sk_i}(\alpha; r)$ if $\sk_i$ is known)
            \item $C = s_j^\alpha t_j^\gamma \bmod N_j$.
\end{itemize}
      Note that $S$ and $C$ are computed using fixed-based multiexponentiations.
        

        \item The prover sends first message $(S, A, C)$, and maintains local (secret) state $(\alpha, \mu, r, \gamma)$.
    \end{itemize}

    \item The verifier chooses $e \gets \pm Q$ and sends $e$ to the prover.

    \item On input $(N_i, K)$, challenge $e$, and local state including $(k, \rho), (\alpha, \mu, r, \gamma)$, the prover computes:
\begin{itemize}
\item    $z_1 = \alpha + e k $
\item    $z_2 = r \cdot \rho^e \bmod N_i $
\item    $z_3 = \gamma + e \mu$.
\end{itemize}
    It then sends $(z_1, z_2, z_3)$ to the verifier.

    \item Given $(N_i, K)$, initial message $(S, A, C)$, challenge $e$, and response $(z_1, z_2, z_3)$, the verifier accepts if and only if all the following are true:
\begin{itemize}
\item $A \oplus (e \odot K) = \enc_{N_i}(z_1; z_2) \bmod N_i^2$ %(the latter is computed as $\enc^\crt_{\sk_i}(z_1; z_2)$ if $\sk_i$ is known)
\item    $s_j^{z_1} t_j^{z_3} = C \cdot S^e \bmod N_j$  
\item    $z_1 \in \pm 2^{\ell + \varepsilon}$.
\end{itemize}
Note the second computation involves a fixed-based multiexponentiation.
\end{enumerate}

\subsubsection{Non-Interactive Version of the Proof}

\begin{itemize}
    \item We deterministically derive a challenge from inputs that include shared $\state$, auxiliary data~$R_j$, the common input~$(N_i, K)$, and the initial protocol message $(S, A, C)$.    
    We write the resulting function as 
    $e = \challengeni{enc}^{L}(\state, R_j, (N_i, K), (S, A, C))$.

%    $\text{seed} = H(\sid \| R \| N_0 \| K \| S \| A \| C) \\ e \stackrel{\text{seed}}{\gets} \pm Q $ 
    
%    \jnote{Would be good to include the prover's ID in the hash. Also $\sid$ (since $s, t, \hat N$ might be shared across many sessions).} \dnote{This is a typo, $\sid$ must be in the hash. Added it. $\sid$ should include prover's ID (it's probably better to call it ctx or CST (context-separation tag)} \jnote{$\sid$ may include the prover's ID among all parties' IDs, but that is not enough: you want the distinguish the prover so the proof can't be replayed by another party} \dnote{Yes I understand that. It's going to work like that: we have $\sid_1$ defined at protocol level (e.g. at presigning) and $\sid_2$ defined at proof level. When proof is invoked, we set $\sid_2 = \sid_1 \| i \| \dots$ where $i$ is party index. I see that it's not taken into account in described presigning below, I will address that (added TBD in presigning section)}

    
    \item The prover generates a proof as follows: first it computes $(S, A, C)$ as described above; then it computes $e = \challengeni{enc}^{L}(\state, R_j, (N_i, K), (S, A, C))$; next, it computes $(z_1, z_2, z_3)$ as described above, using the challenge~$e$. Finally, it outputs the proof $((S, A, C), (z_1, z_2, z_3))$. We write the resulting function as $\proveni{enc}^{L}(\state, R_j, (N_i, K), (k, \rho))$.
    
  %  $\proveni{enc}^{L,R}(\sid, (N_0, K); (k, \rho)) \to ((S, A, C), (z_1, z_2, z_3)):$
   % \begin{enumerate}
    %    \item Prover commits on input:
        
     %   $((S, A, C), (\alpha, \mu, r, \gamma)) = \commit{enc}^{L,R}((N_0, K); (k, \rho))$

      %  \item Derives challenge:

       % $e = \challengeni{enc}^{L,R}(\sid, (N_0, K), (S, A, C))$

       % \item Computes the proof: 

        %$(z_1, z_2, z_3) = \prove{enc}^{L,R}((N_0, K), e; (k, \rho), (\alpha, \mu, r, \gamma))$

%        \item Returns $((S, A, C), (z_1, z_2, z_3))$
 %   \end{enumerate}

    \item A party verifies a proof $\psi=((S, A, C), (z_1, z_2, z_3))$ by first computing \[e = \challengeni{enc}^{L}(\state, R_j, (N_i, K), (S, A, C))\] and then verifying as described above, using the challenge~$e$. We write the resulting function as $\verifyni{enc}^{L}(\state, R_j, (N_i, K), \psi)$.
    
 %   $\verifyni{enc}^{L,R}(\sid, (N_0, K), ((S, A, C), (z_1, z_2, z_3))):$
  %  \begin{enumerate}
   %     \item Derives challenge:

%        $e = \challengeni{enc}^{L,R}(\sid, (N_0, K), (S, A, C))$

 %       \item Asserts $\verify{enc}^{L,R}((N_0, K), (S, A, C), e, (z_1, z_2, z_3))$
  %  \end{enumerate}
\end{itemize}

\subsection{\proof{aff\mbox{-}g}: Paillier Affine Operation with Group Commitment in Range}

%\hat N => N_j
% N_0 => N_j
% N_1 => N_i

We assume the prover and verifier agree on shared state $\state$, auxiliary data\footnote{In \cite{cggmp21}, the auxiliary data is an arbitrary modulus $\hat N$. In the protocol, however, it always holds that $\hat N=N_j$.} $R_j = (N_j, s_j, t_j)$ with $s_j, t_j \in \Z^*_{N_j}$,
an elliptic curve~$\E$ of prime order~$q$ with generator~$G$,
and a security level~$L$.
For this proof, the prover and verifier have common input 
$(N_j, N_i, C, D, Y, X)$ 
where $C, D \in \Z^*_{N_j^2}$, 
$Y \in \Z^*_{N_i^2}$, and
$X \in \E$, and the prover additionally has secret input 
$(x, y, \rho, \rho_y)$ such that 
$x \in \pm 2^\ell$, 
$y \in \pm 2^{\ell'}$, 
$\rho \in \Z^*_{N_j}$, 
$\rho_y \in \Z^*_{N_i}$,  
$D = (x \odot C) \oplus \enc_{N_j}(y; \rho)$, 
$Y = \enc_{N_i}(y; \rho_y)$,
and $X = x \cdot G$.
In all the cases where this proof is used in the protocol, the prover knows the factorization of~$N_i$ (and hence knows~$\sk_i$) and the verifier knows the factorization of~$N_j$ (and hence knows~$\sk_j$).
%    \item Proof guarantees: $x \in \pm 2^{\ell + \varepsilon}$ and $y \in \pm 2^{\ell' + \varepsilon}$
%    \item Implicit constraints: $2^{\ell' + \varepsilon} < N_0/2, N_1/2$

\subsubsection{Interactive Version of the Proof}
\begin{enumerate}
    \item In the first round of the protocol, the prover does the following:
   % $\commit{aff\mbox{-}g}^{L,R}((N_j, N_i, C, D, Y, X); (x, y, \rho, \rho_y)) \to ((A,B_x,B_y,E,S,F,T); (\alpha,\beta,r,r_y,\gamma,\delta,m,\mu))$:
    \begin{itemize}
        \item The prover samples the following values: 

        $\begin{aligned}
            \alpha &\gets \pm 2^{\ell+\varepsilon},   & r   &\gets \Z^*_{N_j}, & \gamma, \delta &\gets \pm (2^{\ell + \varepsilon} \cdot N_j)\\
            \beta &\gets \pm 2^{\ell' + \varepsilon}, & r_y &\gets \Z^*_{N_i}, & m, \mu         &\gets \pm (2^\ell \cdot N_j).
        \end{aligned}$

        \item The prover then computes:
\begin{itemize}
        \item $A = (\alpha \odot C) \oplus \enc_{N_j}(\beta; r)$
          \item  $B_x = \alpha \cdot G $
          \item  $B_y = \enc_{N_i}(\beta; r_y)$ (this is computed as $\enc^\crt_{\sk_i}(\beta; r_y)$ if $\sk_i$ is known)
            \item $E = s_j^\alpha t_j^\gamma \bmod N_j, \; S = s_j^x t_j^m \bmod N_j$ 
            \item $F = s_j^\beta t_j^\delta \bmod N_j, \; T = s_j^y t_j^\mu \bmod N_j$.
        \end{itemize}
        Note that the final two sets of computations are fixed-based multiexponentiations.

        \item The prover sends first message $(A,B_x,B_y,E,S,F,T)$, and maintains local (secret) state $(\alpha, \beta, r, r_y, \gamma, \delta, m, \mu)$.
    \end{itemize}

    \item The verifier chooses $e \leftarrow \pm Q$ and sends $e$ to the prover.

    \item On input $(N_j, N_i, C, D, Y, X)$, the challenge $e$, and local state that includes $(x, y, \rho, \rho_y)$, $(\alpha,\beta,r,r_y,\gamma,\delta,m,\mu)$, the prover computes:

    $\begin{aligned}
        & z_1 = \alpha + e x \\
        & z_2 = \beta + e y \\
        & z_3 = \gamma + e m \\ 
        & z_4 = \delta + e \mu \\
        & w = r \cdot \rho^e \bmod N_j \\
        & w_y = r_y \cdot \rho_y^e \bmod N_i,
    \end{aligned}$
    
    and sends $(z_1, z_2, z_3, z_4, w, w_y)$ to the verifier.

    \item Given $(N_j,N_i,C,D,Y,X)$, initial message $(A,B_x,B_y,E,S,F,T)$, challenge $e$, and response $(z_1, z_2, z_3, z_4, w, w_y)$, the verifier accepts if and only if all the following are true:
    
$\begin{aligned}
    &  A \oplus (e \odot D)  = (z_1 \odot C) \oplus \enc^\crt_{\sk_j}(z_2; w)\bmod N_j^2\\
    &  z_1 \cdot G = B_x + e \cdot X \\
    &  B_y \oplus (e \odot Y)  = \enc_{N_i}(z_2; w_y) \bmod N_i^2\\
    &  s_j^{z_1} t_j^{z_3} = E \cdot S^e \bmod N_j
                                            \\
    &  s_j^{z_2} t_j^{z_4} = F \cdot T^e \bmod N_j \\
    &  z_1 \in \pm 2^{\ell + \varepsilon} \\
    &  z_2 \in \pm 2^{\ell' + \varepsilon}.
    \end{aligned}$
    
Note that two of the above computations involve fixed-base multiexponentiations.
   
\end{enumerate}

\subsubsection{Non-Interactive Version of the Proof}
\begin{itemize}
    \item We deterministically derive a challenge from inputs that include shared $\state$, the auxiliary data~$R_j$, the common input~$(N_j,N_i,C,D,Y,X)$, and the initial protocol message $(A,B_x,B_y,E,S,F,T)$.    
We write the resulting function as \[e = \challengeni{aff\mbox{-}g}^{L}(\state,R_j,(N_j,N_i,C,D,Y,X),(A,B_x,B_y,E,S,F,T)).\]

    \item The prover generates a proof as follows: it computes its initial message $(A,B_x,B_y,E,S,F,T)$ as described above; then it computes \[e = \challengeni{aff\mbox{-}g}^{L}(\state,R_j,(N_j,N_i,C,D,Y,X),(A,B_x,B_y,E,S,F,T));\] next, it computes $(z_1,z_2,z_3,z_4,w,w_y)$ as described above, using the challenge~$e$. Finally, it outputs the proof $((A,B_x,B_y,E,S,F,T), (z_1,z_2,z_3,z_4,w,w_y))$. We write the resulting function as $\proveni{aff\mbox{-}g}^{\E,L}(\state, R_j, (N_j, N_i, C, D, Y, C); (x,y,\rho,\rho_y))$. 
  
    \item A party verifies a proof $\psi=((A,B_x,B_y,E,S,F,T), (z_1,z_2,z_3,z_4,w,w_y))$ by first computing 
    \[e = \challengeni{aff\mbox{-}g}^{L}(\state,R_j,(N_j,N_i,C,D,Y,X),(A,B_x,B_y,E,S,F,T))\]
    and then verifying as described above, using the challenge~$e$. We write the resulting function as $\verifyni{aff\mbox{-}g}^{\E,L}(\state,R_j, (N_j,N_i,C,D,Y,X), \psi)$.
\end{itemize}

\subsection{$\proof{mod}$: Paillier-Blum Modulus}
The prover and verifier agree on shared state $\state$ and a security level~$L$ (which determines~$m$). For this proof, the prover and verifier have common input~$N$, and the prover additionally has as secret input primes $p, q=3 \bmod 4$ such that~$N=pq$.

\subsubsection{Interactive Version of the Proof}

\begin{enumerate} 
\item In the first round of the protocol, the prover samples uniform 
$w \in \mathbb{Z}_N$
    with Jacobi symbol $\left( \frac{w}{N} \right) = -1$. It sends $w$ to the verifier and maintains local state~$w$.

  \item The verifier chooses uniform $y_i \in \mathbb{Z}_N$ for $i=1,\ldots, m$.

  \item Given $N$, the challenge $y_1, \ldots, y_m$, and local state that includes $p, q, w$, the prover does the following for $i=1,\ldots,m$:
    \begin{enumerate}
      \item Compute $a_i, b_i \in \{0,1\}$ such that $y'_i = (-1)^{a_i} w^{b_i} y_i \bmod N$ is a quadratic residue  modulo~$N$.
      \item Let $x_i$ be the principal\footnote{This means that $x_i$ is itself a quadratic residue.} 4th root of $y'_i$ modulo~$N$.
      \item Compute $N' = N^{-1} \bmod \phi(N)$ and set $z_i = y_i ^ {N'} \bmod N$.
    \end{enumerate}
Send $\{(x_i, a_i, b_i, z_i)\}_{i=1}^m$ to the verifier. 
    
  \item Given $N$, initial message $w$, challenge $\{y_i\}_{i=1}^m$, and response 
  $\{(x_i, a_i, b_i, z_i)\}_{i=1}^m$, the verifier accepts if and only if all the following are true:
    \begin{enumerate}
      \item $N$ is an odd non-prime.
      \item For $i \in \{1, \ldots, m\}$: \begin{itemize}
          \item $z_i^N = y_i \bmod N$.
          \item $x_i^4 = (-1)^{a_i} w^{b_i} y_i \bmod N$.
      \end{itemize} 
    \end{enumerate}

\end{enumerate}

\subsubsection{Non-Interactive Version of the Proof}

\begin{itemize}
  \item We deterministically derive a challenge from inputs that include shared $\state$, the common input~$N$, and the initial protocol message~$w$.   
  We write the resulting function as $(y_1, \ldots, y_m)=\challengeni{mod}^L(\state, N, w)$.

  \item The prover generates a proof by first computing its initial message $w$ as described above; then it computes 
        $(y_1, \ldots, y_m) = \challengeni{mod}^L(\state, N, w)$; next, it computes $\{(x_i, a_i, b_i, z_i)\}_{i=1}^m$ as described above, using the challenge $\{y_i\}_{i=1}^m$. It outputs the proof $(w, \{(x_i, a_i, b_i, z_i)\}_{i=1}^m)$. We write the resulting function as $\proveni{mod}^L((\state, N), (p, q))$.

  \item A party verifies a proof  $(w, \{(x_i, a_i, b_i, z_i)\}_{i=1}^m)$ by first computing \[(y_1, \ldots, y_m) = \challengeni{mod}^L(\state, N, w)\] and then verifying as described above, using the challenge~$\{y_i\}_{i=1}^m$.
\end{itemize}

\subsection{$\proof{prm}$: Ring-Pedersen Parameters}
The prover and verifier agree on shared state $\state$ and security level~$L$ (which determines~$m$). 
For this proof, the prover and verifier have common input $(N, s, t)$ with $s, t \in {\mathbb Z}_N^*$, and the prover additionally has secret input $\lambda$ such that $s=t^\lambda \bmod N$, along with the factorization of~$N$.

\subsubsection{Interactive Version of the Proof}
\begin{enumerate} 
  \item In the first round of the protocol, the prover first does the following for $i =1, \ldots, m$:
    \begin{itemize}
      \item The prover samples $a_i \gets \Z_{\phi(N)}$.
      \item The prover computes $A_i = t^{a_i} \bmod N$.
    \end{itemize}
The prover then sends first message $\{A_i\}_{i=1}^m$ and maintains local (secret) state $\{a_i\}_{i=1}^m$.

  \item For $i=1, \ldots, m$, the verifier chooses $e_i \leftarrow \{0,1\}$, and sends $\{e_i\}_{i=1}^m$ to the prover.

  \item On input $N, s, t$, the challenge $\{e_i\}_{i=1}^m$, and local state including $\phi(N), \lambda$, and the $\{a_i\}_{i=1}^m$, for $i=1, \ldots, m$ the prover computes $z_i= a_i + e_i \cdot \lambda \bmod \phi(N)$. It sends $\{z_i\}_{i=1}^m$ to the verifier.

  \item Given $N, s, t$, initial message $\{A_i\}_{i=1}^m$, challenge $\{e_i\}_{i=1}^m$, and response $\{z_i\}_{i=1}^m$, the verifier accepts if and only if 
$t^{z_i} = A_i s^{e_i} \bmod N$ for $i=1, \ldots, m$. 
\end{enumerate}

\subsubsection{Non-Interactive Version of the Proof}
\begin{itemize}
  \item We deterministically derive a challenge from inputs that include shared $\state$, the common input~$(N,s,t)$, and the initial protocol message~$\{A_i\}_{i=1}^m$.   
  We write the resulting function as
  $\{e_i\}_{i=1}^m = \challengeni{prm}^L(\state, N, s, t, \{A_i\}_{i=1}^m)$.

  \item The prover generates a proof as follows: first, it computes its initial message $\{A_i\}_{i=1}^m$ as described above; then it computes $\{e_i\}_{i=1}^m = \challengeni{prm}^L(\state, N, s, t, \{A_i\}_{i=1}^m)$; next, it computes $\{z_i\}_{i=1}^m$ as described above, using the challenge $\{e_i\}_{i=1}^m$. Finally, it outputs the proof $(\{A_i\}_{i=1}^m, \{z_i\}_{i=1}^m)$. We write the resulting function as $\proveni{prm}^L(\state, (N, s, t), (\phi, \lambda))$.
  
  \item A party verifies a proof $\psi=(\{A_i\}_{i=1}^m, \{z_i\}_{i=1}^m)$ for $(N, s, t)$ by  setting \[\{e_i\}_{i=1}^m = \challengeni{prm}^L(\state, N, s, t, \{A_i\}_{i=1}^m)\] and then verifying as described above, using the challenge~$\{e_i\}_{i=1}^m$. We write the resulting function as $\verifyni{prm}^L(\state, (N, s, t), \psi)$.
  \end{itemize}

\subsection{$\proof{log*}$: Group Element vs.\ Paillier Encryption in Range}
The prover and verifier agree on shared state $\state$, auxiliary data $R_j=(N_j, s_j, t_j)$ with $s_j, t_j \in {\mathbb Z}^*_{N_j}$, 
an elliptic curve~$\E$ of prime order~$q$ with generator~$G$,
and a security level~$L$. For this proof, the prover and verifier have common input $(N_i, C, X, B)$ with $C \in {\mathbb Z}^*_{N_i^2}$ and $X, B \in \E$, and the prover additionally has secret input $(x, \rho)$ such that $x \in \pm 2^\ell$, $C=\enc_{N_i}(x; \rho)$, and $X = x \cdot B$.
In all the cases where this proof is used in the protocol, the prover knows the factorization of $N_i$ (and hence knows~$\sk_i$) and the verifier knows the factorization of $N_j$ (and hence knows~$\sk_j$).

%    \item Proof guarantees: $x \in \pm 2^{\ell + \varepsilon}$
 %   \item Implicit constraints: $2^{\ell + \varepsilon} < N_0/2$


\subsubsection{Interactive Version of the Proof}

\begin{enumerate}
\item In the first round of the protocol, the prover does the following:
    \begin{itemize}
        \item The prover samples the following values:

        $\begin{aligned}
            \alpha &\gets \pm 2^{\ell + \varepsilon} \\
            \mu &\gets \pm (2^\ell \cdot N_j) \\
            r &\gets \Z^*_{N_0} \\
            \gamma &\gets \pm (2^{\ell + \varepsilon} \cdot N_j).
        \end{aligned}$

        \item The prover then computes:
\begin{itemize}
\item        $
            S = s_j^x t_j^\mu \bmod N_j$ 
\item            $A = \enc_{N_i}(\alpha; r)$ (this is computed as $\enc^\crt_{\sk_i}(\alpha; r)$ when $\sk_i$ is known) 
\item            $Y = \alpha \cdot B$ 
\item            $D = s_j^\alpha t_j^\gamma \bmod N_j.$ 
\end{itemize}
Note that $S$ and $D$ are computed using fixed-base multiexponentiations.
        \item The prover sends first message $(S,A,Y,D)$ and maintains local (secret) state $(\alpha,\mu,r,\gamma)$.
    \end{itemize}

    \item The verifier chooses $e \leftarrow \pm Q$ and sends $e$ to the prover.

    \item On input $(N_i, C, X, B)$, the challenge $e$, and local state that includes $(x,\rho), (\alpha,\mu,r,\gamma)$, the prover computes:

    $\begin{aligned}
        z_1 &= \alpha + e x \\
        z_2 &= r \cdot \rho^e \bmod N_i \\
        z_3 &= \gamma + e \mu,
    \end{aligned}$
    
and sends $(z_1, z_2, z_3)$ to the verifier.

    \item Given $(N_i,C,X,B)$, initial message $(S,A,Y,D)$, challenge $e$, and response $(z_1,z_2,z_3)$, the verifier accepts if and only if the following are true:

    $\enc_{N_i}(z_1; z_2)  = A \oplus (e \odot C) \bmod N_i^2\\
    z_1 \cdot B = Y + e \cdot X \\
    s_j^{z_1} t_j^{z_3} = D \cdot S^e \bmod N_j$  \\
    $z_1 \in \pm 2^{\ell + \varepsilon}$.
\end{enumerate}

\subsubsection{Non-Interactive Version of the Proof}

\begin{itemize}
    \item We deterministically derive a challenge from inputs that include shared $\state$, the auxiliary data~$R_j$, the common input~$(N_i,C,X, B)$, and the initial protocol message~$(S,A,Y,D)$.   
    We write the resulting function as 
    $e=\challengeni{log*}^{\E,L}(\state, R_j, (N_i,C,X, B), (S,A,Y,D))$.

 %   $\text{seed} = H(\sid \| \E \| L \| R \| N_0 \| C \| X \| S \| A \| Y \| D) \\ e \stackrel{\text{seed}}{\gets} \pm Q$ 

    \item The prover generates a proof by first computing its initial message $(S, A, Y, D)$ as described above, then computing $e = \challengeni{log*}^{\E,L}(\state, R_j,(N_i,C,X, B), (S,A,Y,D))$, and next computing $(z_1,z_2,z_3)$ as described above, using challenge~$e$. It outputs the proof $((S, A, Y, D)$, $(z_1, z_2, z_3))$. We write the resulting function as $\proveni{log*}^{\E,L}(\state,R_j, (N_i, C, X, B); (x,\rho))$.

    \item A party verifies a proof $\psi-((S, A, Y, D), (z_1, z_2, z_3))$ by first computing \[e = \challengeni{log*}^{\E,L}(\state, R_j, (N_i,C,X, B), (S,A,Y,D))\] and then verifying as described above, using the challenge~$e$. We write the resulting function as $\verifyni{log*}^{\E,L}(\state, R_j, (N_i, C, X, B), \psi)$.
\end{itemize}

\subsection{$\proof{fac}$: No Small Factor Proof}
The prover and verifier agree on shared state $\state$, auxiliary data $R_j=(N_j, s_j, t_j)$ with $s_j, t_j \in {\mathbb Z}_{N_j}^*$, and a security level~$L$. 
For this proof, the prover and verifier have common input~$N_i$, and the prover additionally has primes $2^\ell < p, q < \pm 2^\ell \cdot \sqrt{N_i}$ with $N_i=pq$. 
In all the cases where this proof is used in the protocol, the verifier knows the factorization of $N_j$ (and hence knows~$\sk_j$).

%  \item Proof guarantees: $p_0, q_0 > 2^\ell$
%  \item Implicit constraints:   $2^{2\ell + \varepsilon} \approx \sqrt{N_0}$ \jnote{Would be good to formalize exactly what this means} \nnote{I aped it from the paper. It seems they mention it to reconcile the fact that sometimes they mention this proof with $2^l$ as the lower limit, and sometimes with $2^l \sqrt{N_0}$ as the upper - this way they are the same (not taking $\epsilon$ into account)}


\subsubsection{Interactive Version of the Proof}
\begin{enumerate}
  \item In the first round of the protocol, the prover does the following:

    \begin{itemize}
      \item The prover samples the following values:

        $\alpha, \beta \gets \pm \left(2^{\ell + \varepsilon} \sqrt{N_i}\right)$ \\
        $\mu, \nu \gets \pm \left(2^{\ell} N_j\right)$ \\
        $\sigma \gets \pm \left(2^{\ell} N_i N_j\right)$ \\
        $r \gets \pm \left(2^{\ell + \varepsilon} N_i N_j\right)$ \\
        $x, y \gets \pm \left(2^{\ell + \varepsilon} N_j\right)$.

      \item The prover then computes:
\begin{itemize}
\item        $P = s_j^{p} t_j^\mu \bmod N_j$ 
\item        $Q = s_j^{q} t_j^\nu \bmod N_j$ 
\item        $A = s_j^\alpha t_j^x \bmod N_j$ 
\item        $B = s_j^\beta t_j^y \bmod N_j$ 
\item        $T = Q^\alpha t_j^r \bmod N_j$.
 \end{itemize}
 Note that $P,Q,A,B$ are computed using fixed-base multiexponentiations. %(It is also possible to express computation of $T$ as a fixed-base multiexponentiation $T=(s^qt^\nu)^\alpha \cdot t^r = s^{q\alpha}t^{\nu\alpha+r} \bmod N_j$ and then use multiexp preprocessing, but I'm not sure this will be a net win.)
      \item The prover sends the first message  $(P, Q, A, B, T, \sigma)$ 
        and also maintains local (secret) state $(\alpha, \beta, \mu, \nu, r, x, y)$.
    \end{itemize}

  \item The verifier chooses $e \leftarrow \pm Q$ and sends $e$ to the prover.

  \item On input $N_i$, the challenge $e$, and local state that includes $(p, q)$, $\sigma$, and 
  $(\alpha, \beta, \mu, \nu, r, x, y)$, the prover computes: \\
    $z_1 = \alpha + ep$ \\
    $z_2 = \beta + eq$ \\
    $w_1 = x + e\mu$ \\
    $w_2 = y + e\nu$ \\
    $v = r + e \cdot (\sigma - \nu p)$,

    and sends $(z_1, z_2, w_1, w_2, v)$ to the verifier.

  \item Given $N_i$, initial message $(P, Q, A, B, T, \sigma)$, challenge $e$, and response $(z_1, z_2, w_1, w_2, v)$, the verifier accepts if and only if the following are true: \begin{itemize}
\item    $s_j^{z_1} t_j^{w_1} = A \cdot P^e \bmod N_j$ 
\item    $s_j^{z_2} t_j^{w_2} = B \cdot Q^e \bmod N_j$ 
\item    $Q^{z_1} t_j^{v} = T \cdot (s_j^{N_i} t_j^\sigma)^e \bmod N_j$ 
\item    $z_1 \in \pm \left(2^{\ell + \varepsilon} \sqrt{N_i}\right)$ 
\item    $z_2 \in \pm \left(2^{\ell + \varepsilon} \sqrt{N_i}\right)$.
\end{itemize}
Note that the 1st and 2nd checks involve fixed-base multiexponentiations.

\end{enumerate}

\subsubsection{Non-Interactive Version of the Proof}

\begin{itemize}
  \item We deterministically derive a challenge from inputs that include shared $\state$, the auxiliary data~$R_j$, the common input~$N_i$, and the initial protocol message~$(P, Q, A, B, T, \sigma)$.
  We write the resulting function as
  $e=\challengeni{fac}^{L} (\state, R_j, N_i, (P, Q, A, B, T, \sigma))$.

 

  \item A proof is computed as follows:  compute initial message $(P, Q, A, B, T, \sigma)$ as described above; compute $e=\challengeni{fac}^{L} (\state, R_j, N_i, (P, Q, A, B, T, \sigma))$; next compute $(z_1, z_2, w_1, w_2, v)$ as described above, using challenge~$e$. Output the proof $((P, Q, A, B, T, \sigma), (z_1, z_2, w_1, w_2, v))$. We write the resulting function as $\proveni{fac}^{L}(\state, R_j, N_i, (p_i, q_i))$.

  \item A party verifies a proof $\phi=((P, Q, A, B, T, \sigma), (z_1, z_2, w_1, w_2, v))$ by first computing \[e=\challengeni{fac}^{L} (\state, R_j, N_i, (P, Q, A, B, T, \sigma))\] and then verifying as described above, using the challenge~$e$. We write the resulting function as $\verifyni{fac}^{L}(\state, R_j, N_i, \phi)$.
\end{itemize}


\subsection{\proof{sch}: Schnorr Proof of Knowledge}
We describe the standard Schnorr proof of knowledge, and also set up notation that we will use in what follows.
\begin{itemize}
    \item $\commit{sch}() \to (A; \alpha)$ \\
    $\alpha \gets \Z_q \\
    A = \alpha \cdot G$ \\
    return $(\alpha, A)$

    \item $\challenge{sch}() \to e$ \\
    return $e \gets \Z_q$

    \item $\prove{sch}(\alpha, e, x) \to z$ \\
    return $z = \alpha + e x \bmod q$

    \item $\verify{sch}(z, A, e, X)$ \\
    accept iff $z \cdot G = A + e \cdot X$.
\end{itemize}


\section{Threshold Protocols}

In this section we describe the various threshold protocols we have implemented. 
Overall, we have the following protocols:
\begin{enumerate}
    \item When a ``signing cluster'' is initialized, the signers in that cluster run a provisioning protocol in which they each generate auxiliary information. That protocol is described in Section~\ref{section:aux_data_gen}. 

    \item When key generation is requested in an initialized cluster, the signers in that cluster run a distributed key-generation protocol. 
    We have implemented protocols for both $n$-out-of-$n$ key generation (cf.\ Section~\ref{sec:n-of-n-dkg}), as well as $t$-out-of-$n$ key  generation (cf.\ Section~\ref{sec:t-of-n-dkg}).

    \item A threshold of signers who have already generated a key can compute \emph{presignatures} with respect to that key. Protocols for doing that, in both the ``non-threshold'' (i.e., $n$-out-of-$n$) and ``threshold'' (i.e., $t$-out-of-$n$) cases, are described in Section~\ref{sec:presigning}.

    \item If a presignature has already been generated by some threshold of signers with respect to some key, those signers can non-interactively compute a signature on a given message~$m$ using the presignature they have computed. See Section~\ref{sec:signing}.
\end{enumerate}

\ignore{We have also implemented a protocol for $n$-out-of-$n$ key refresh that refreshes the shares associated with some key.
Since we have not yet implemented a $t$-out-of-$n$ version of key refresh, and do not currently use key refresh in production, we have deferred a description of that protocol to
 Appendix~\ref{sec:refresh}.
(Note that when key refresh is used in some signing cluster, one should probably do a key refresh for \emph{all} the keys stored by that  cluster, and should also re-run the provisioning protocol for that cluster.)
}

\subsection{Provisioning Protocol}
\label{section:aux_data_gen}

This protocol is run to generate auxiliary information for each signer in a cluster.  

\begin{description}
  \item[\textbf{Input.}]
    Party index $i$,
    context separation string $\sid$,
    security level $L$.

  \item[\textbf{Round 1.}] \
    \begin{itemize}
    \item Generate $4\kappa$-bit safe primes $p_i, q_i$ using the algorithm from Section~\ref{sec:safe-prime}.
      \item Compute $N_i = p_iq_i$ and $\phi = (p_i - 1)(q_i - 1)$, and create a Paillier decryption key $\sk_i$ using those values.
     \item Sample  $r \gets \Z_{N_i}^{*}$ and $\lambda \gets \Z_\phi$,
        and compute $t_i = r_i^2 \bmod N_i$ and $s_i = t_i^\lambda \bmod N_i$.
      \item Compute $\hat{\psi}_i = \proveni{prm}^L((\sid, i), (N_i, s_i, t_i), (\phi, \lambda))$.
      \item Sample $\rho_i, u_i \leftarrow \{0,1\}^\kappa$, 
        and compute $V_i =
        H(\Encode{hash\_com}(\sid, n, i, N_i, s_i, t_i, \hat{\psi}_i, \rho_i, u_i))$.
      \item Send $V_i$ to all parties.
    \end{itemize}
    

  \item[\textbf{Round 2.}] \
    \begin{itemize}
      \item Receive $V_j$ from all parties.
      \item ({\bf Reliability check.}) Optionally, if the reliability check is enabled:
    \begin{itemize}
        \item 
        Compute $h_i = H(\Encode{echo}(V_0, \dots, V_{n-1}))$ and 
            send $h_i$ to all parties.
        
        
        \item Upon receiving $h_j$ from all parties, abort if $h_i \neq h_j$ for some $j \in [n]$.
    \end{itemize}
      \item Send $(N_i, s_i, t_i, \hat{\psi}_i, \rho_i, u_i)$ to all parties.
    \end{itemize}

  \item[\textbf{Round 3.}] \

    \begin{itemize}
      \item Receive $(N_j, s_j, t_j, \hat{\psi}_j, \rho_j, u_j)$
        from all parties.
      \item For all $j \in [n]$, set $R_j = (N_j, s_j, t_j)$; let
       $\vec{R} = (R_j)_{j \in [n]}$.

      \item For $j \neq i$:
        \begin{itemize}
          \item Assert $V_j =
            H(\Encode{hash\_com}(\sid, n, j, N_j, s_j, t_j, \hat{\psi}_j, \rho_j, u_j))$.
          \item Assert $N_j$ is at least $8 \cdot \kappa - 1$ bits in length
          \item Assert $\verifyni{prm}^L((\sid, j), (N_j, s_j, t_j), \hat{\psi}_j)$.
          \item Construct Paillier encryption key from~$N_j$.
        \end{itemize}
\item Compute $\rho=\bigoplus_j \rho_j$.
      \item Compute $\psi_i = \proveni{mod}^L((\sid, i, \rho), N_i, (p_i, q_i))$.
      \item For $j\neq i$ do:
        \begin{itemize}
          \item Compute $\phi_i^j = \proveni{fac}^{L}((\sid, i, \rho), R_j, N_i, (p_i, q_i))$.
          \item Send $(\psi_i, \phi_i^j)$ to $P_j$.
        \end{itemize}

    \end{itemize}

  \item[\textbf{Output.}] \

    \begin{itemize}
      \item Receive $(\psi_j, \phi_j^i)$ from all parties.
      \item For $j \neq i$ do: 
        \begin{itemize}
          \item Assert $\verifyni{mod}^L((\sid, j, \rho), N_j, \psi_j)$.
          \item Assert $\verifyni{fac}^{L}((\sid, j, \rho), R_i, N_j, \phi_j^i)$.
        \end{itemize}

      \item For $j \in [n]$ (including $j=i$), precompute a fixed-based multiexponentiation table $T_j$ as described in Section~\ref{sec:multiexp}. Let $\vec{T} = (T_j)_{j \in [n]}$.
      \item Return $(p_i, q_i, \vec{R}, \vec{T})$.
    \end{itemize}

\end{description}


\subsection{Distributed Key Generation}
We implement two versions of distributed key generation. One generates a key along with an $n$-out-of-$n$ additive sharing of that key, and the other generates a key along with a $t$-out-of-$n$ Shamir secret sharing of that key. Note that only the former protocol is described in~\cite{cggmp21}.

\subsubsection{Non-Threshold (i.e., $n$-out-of-$n$) Key Generation}
\label{sec:n-of-n-dkg}

This protocol is based on~\cite[Figure~5]{cggmp21}, but we have added the option to replace the broadcast channel with a reliable broadcast subroutine, and we added optional support of HD-wallets.

\begin{description}
\item[\bf Input.]
    Party index $i$,
    number of signers $n$,
    context separation string \sid,
    security level~$L$,
    curve $\E$ with generator $G$ of prime order $q$.

\item[\bf Round 1.] \
\begin{itemize}
    \item Sample $x_i \gets \Z_q$, 
          and set $X_i = x_i \cdot G$.
    \item Sample $\rid_i \gets \{0,1\}^\kappa$.
    \item Compute $(A_i; \tau_i) = \commit{sch}()$.
    \item {\bf (HD-wallets.)} 
    \begin{itemize}
      \item If HD-wallets support enabled, sample local chain code contribution $c_i \gets \{0,1\}^{256}$ (32-bytes string)
      \item Otherwise, set $c_i = \bot$
    \end{itemize}
    \item Sample $u_i \gets \{0,1\}^\kappa$
          and set $V_i = H(\Encode{hash\_com}(\sid, n, i, \rid_i, X_i, A_i, u_i, c_i))$.
    \item Send $V_i$ to all parties.
\end{itemize}

\item[\bf Round 2.] Upon receiving $V_j$ from all parties:
\begin{itemize}
    \item {\bf (Reliability check.)} Optionally, if the reliability check is enabled:
\begin{itemize}
    \item Compute $h_i = H(\Encode{echo}(V_0, \dots, V_{n-1}))$ and
        send $h_i$ to all parties.
    
    \item Upon receiving $h_j$ from all other parties: abort if $h_i \neq h_j$ for some $j\in [n]$.
\end{itemize}
    \item Send $(\rid_i, X_i, A_i, u_i, c_i)$ to all parties.
    \end{itemize}

\item[\bf Round 3.] 
    Upon receiving $(\rid_j, X_j, A_j, u_j, c_j)$ from all other parties:
    \begin{itemize}
    \item Abort if $V_j \neq H(\Encode{hash\_com}(\sid, n, j, \rid_j, X_j, A_j, u_j, c_j))$
          for some $j \in [n]$.
    \item Set $\rid = \bigoplus_j \rid_j$.
    \item {\bf (HD-wallets.)} If HD-wallets support enabled:
    \begin{itemize}
      \item Set chain code $c = \bigoplus_j c_j$
    \end{itemize}
    \item Set $e_i = \challengeni{sch}(\sid, i, \rid, X_i, A_i)$
          and compute $\psi_i = \prove{sch}(\tau_i, e_i, x_i)$.
    \item Send $\psi_i$ to all parties.
    \end{itemize}

\item[\bf Output.] 
    Upon receiving $\psi_j$ from all other parties:
    \begin{itemize}
    \item 
        For all $j \ne i$:
        \begin{itemize}
            \item Set $e_j = \challengeni{sch}(\sid, j, \rid, X_j, A_j)$.
            \item Assert $\verify{sch}(\psi_j, A_j, e_j, X_j)$.
        \end{itemize}

    \item Set $X = \sum_j X_j$,
              $\vec X = (X_j)_{j\in[n]}$.

    \item Output $(X, x_i, \vec X, c)$.
    \end{itemize}

\end{description}

\subsubsection{Threshold (i.e., $t$-out-of-$n$) Key Generation}
\label{sec:t-of-n-dkg}

\begin{description}

  \item[\textbf{Input.}]
    Party index $i$,
    threshold parameter $t$,
    number of signers $n$,
    context separation string $\sid$,
    security level~$L$,
    curve $E$ with generator $G$ of prime order $q$.

  \item[\textbf{Round 1.}] \
    \begin{itemize}
      \item Sample $s_{i,0}, \ldots, s_{i,{t-1}} \gets \Z_q$. 
            Set $\vec S_i = (s_{i,k} \cdot G)_{k \in [t]}$.
            Let $f_i(x) = \sum_{k \in [t]} s_{i,k} \cdot x^k$ and $F_i(x) = f(x) \cdot G$.
      \item Compute $\sigma_{i,j} = f_i(j + 1)$ for all $j \in [n]$.
      \item Sample $\rid_i \gets \{0,1\}^\kappa$.
      \item Compute $(A_i, \tau_i) \gets \commit{sch}()$. % $(r_i, h_i) \gets \commit{sch}()$
      \item {\bf (HD-wallets.)} 
      \begin{itemize}
        \item If HD-wallets support enabled, sample local chain code contribution $c_i \gets \{0,1\}^{128}$ (32-bytes string)
        \item Otherwise, set $c_i = \bot$
      \end{itemize}

      \item Sample $u_i \gets \{0,1\}^\kappa$
        and compute $V_i = H(\Encode{hash\_com}(\sid, n, i, t, \rid_i, \vec{S_i}, A_i, u_i, c_i))$.
      \item Send $V_i$ to all parties.
    \end{itemize}


  \item[\textbf{Round 2.}] 
 Upon receiving $V_j$ from all parties:
    \begin{itemize}
      \item {\bf (Reliability check.)} Optionally, if the reliability check is enabled:
    \begin{itemize}
        \item 
        Compute $h_i = H(\Encode{echo}(V_0, \cdots, V_{n-1}))$, and send $h_i$ to all parties.
        
        \item Upon receiving $h_j$ from all  parties: abort if 
        $h_i \neq h_j$ for some $j \in [n]$.
    \end{itemize}
      \item Send $(\rid_i, \vec{S}_i, A_i, u_i, c_i)$ to all parties.
      \item For all $j \neq i$, send $\sigma_{i,j}$ to $P_j$ via private channel.
    \end{itemize}

  \item[\textbf{Round 3.}] 
    Upon receiving $(\rid_j, \vec{S}_j, A_j, u_j, c_j)$ and $\sigma_{j,i}$ from all parties:
    \begin{itemize}
      \item For each party $j \neq i$:
        \begin{itemize}
        \item Check that $\vec{S}_j$ has length $t$.
        \item Assert $V_j = H(\Encode{hash\_com}(\sid, n, j, t, \rid_j, \vec{S}_j, A_j, u_j, c_j))$.
        \item Define 
          $F_j(x) = \sum_{k \in [t]} x^k \cdot S_{j,k}$.
        \item Assert $\sigma_{j,i} \cdot G  = F_j(i + 1)$.
        \end{itemize}
        \item Compute $\rid = \bigoplus_{j \in [n]} \rid_j$.
        \item {\bf (HD-wallets.)} If HD-wallets support enabled:
        \begin{itemize}
          \item Set chain code $c = \bigoplus_{j \in [n]} c_j$
        \end{itemize}
        \item Let $F(x) = \sum_{k \in [t]} x^k \cdot \left(\sum_{j\in [n]} S_{j,k}\right) = \sum_{j \in [n]} F_j(x)$. 
        \item   For $j \in [n]$, compute
            $X_j = F(j + 1)$. 
            Let $\vec{X} = (X_j)_{j \in [n]}$.
      \item Compute $x_i = \sum_{j \in [n]} \sigma_{j,i}$.
      \item Compute challenge $e_i = \challengeni{sch}(\sid, i, \rid, X_i, A_i)$.
      \item Compute Schnorr proof $\psi_i = \prove{sch}(\tau_i, e_i, \sigma_i)$.
      \item Send $\psi_i$ to all parties.
    \end{itemize}

  \item[\textbf{Output.}] 
  Upon receiving $\psi_j$ from all parties:
    \begin{itemize}
      \item For $j \neq i$:
      set $e_j = \challengeni{sch}(\sid, j, \rid, X_j, A_j)$ and 
      assert $\verify{sch}(\psi_j, A_j, e_j, X_j)$.
      \item Compute $Y = \sum_{j \in [n]} S_{j,0}$.
      \item Create identity mapping $I : [n] \to \Z_q \setminus 0$,
        $I(i) = i + 1$.
      \item Return $(Y, x_i, \vec{X}, I, c)$.
    \end{itemize}

\end{description}



\subsection{Presigning}
\label{sec:presigning}

We implemented both non-threshold and threshold versions of presigning. 
The non-threshold version assumes the $n$ parties running the protocol have additive shares of the private key. 
The threshold version of the protocol, which is not described in~\cite{cggmp21}, first maps the key shares (which are a $t$-out-of-$n$ Shamir sharing of the private key) to a $t$-out-of-$t$ sharing of the key using Lagrange interpolation and then runs the non-threshold protocol among the $t$ parties.



\subsubsection{Non-Threshold ($n$-out-of-$n$) Presigning} \label{section:orig-presigning}

The following protocol is based on \cite[Figure~7]{cggmp21}, although we have corrected some typos and eliminated some extraneous parts. (In particular, we do not have identifiable abort.)

\begin{description}
    \item[\textbf{Input.}] Number of parties $n$, party index $i\in [n]$, secret share $x_i$, list of signers' public-key shares $\vec X = \{X_j\}_{j \in [n]}$, 
    %public key $\pk = \sum_{j\in[n]} X_j$, 
    Paillier private key $\sk_i$, list of signers' auxiliary data $\vec R = ((s_j, t_j, N_j))_{j\in[n]}$, context separation string $\sid$, security level~$L$, elliptic curve $\E$ with generator~$G$.

    \item[\textbf{Round 1.}] \
    \begin{itemize}
        \item Sample $k_i, \gamma_i \gets \Z_q$, $\rho_i, v_i \gets \Z^*_{N_i}$, and set $G_i = \enc^\crt_{\sk_i}(\gamma_i; v_i)$,
            $K_i = \enc^\crt_{\sk_i}(k_i; \rho_i)$. 
        \item For $j \ne i$ compute $\psi^0_{j,i} = \proveni{enc}^{L}((\sid, i), R_j, (N_i, K_i), (k_i, \rho_i))$. 
        \item Send $(K_i, G_i)$ to all parties, and for $j \neq i$ send $\psi^0_{j,i}$ to $P_j$.
    \end{itemize}


    \item[\textbf{Round 2.}] Upon receiving $(K_j, G_j, \psi^0_{i,j})$ from all parties, do:
    \begin{itemize}
     \item {\bf (Reliability check.)} Optionally, if the reliability check is enabled:
    \begin{itemize}
        \item Compute $h_i = H(\Encode{echo}(K_0, G_0, \cdots, K_{n-1}, G_{n-1}))$
        and 
        send $h_i$ to all parties.
        
        \item Upon receiving $h_j$ from all parties, abort if 
        $h_i \neq h_j$ for some $j \in [n]$.
    \end{itemize}
        \item For $j \neq i$, assert $\verifyni{enc}^{L}((\sid, j), R_i, (N_j, K_j), \psi^0_{i,j})$. 
        
        \item Compute $\Gamma_i = \gamma_i \cdot G$.
        \item For $j \neq i$ do:
        \begin{itemize}
            \item Sample $r_{i,j}, \hat{r}_{i,j} \gets \Z_{N_i}$,
                $s_{i,j}, \hat{s}_{i,j} \gets \Z_{N_j}$, and
                $\beta_{i,j}, \hat{\beta}_{i,j} \gets \pm 2^{\ell'}$.
            \item Compute $D_{j,i} = (\gamma_i \odot K_j) \oplus \enci{j}(-\beta_{i,j}; s_{i,j})$
                and $F_{j,i} = \enc^\crt_{\sk_i}(-\beta_{i,j}; r_{i,j})$.
            \item Compute $\hat{D}_{j,i} = (x_i \odot K_j) \oplus \enci{j}(-\hat{\beta}_{i,j}; \hat{s}_{i,j})$
                and $\hat{F}_{j,i} = \enc^\crt_{\sk_i}(-\hat{\beta}_{i,j}; \hat{r}_{i,j})$.

            \item Compute $\psi_{j,i} = \proveni{aff\mbox{-}g}^{\E,L}((\sid, i), R_j, (N_j, N_i, K_j, D_{j,i}, F_{j,i}, \Gamma_i); (\gamma_i, -\beta_{i,j}, s_{i,j}, r_{i,j}))$ and
            $\hat{\psi}_{j,i} = \proveni{aff\mbox{-}g}^{\E,L}((\sid, i), R_j, (N_j, N_i, K_j, \hat{D}_{j,i}, \hat{F}_{j,i}, X_i); (x_i, -\hat{\beta}_{i,j}, \hat{s}_{i,j}, \hat{r}_{i,j}))$. 
            
            %\jnote{To consider later: this and the previous could possibly be combined for better efficiency}
            \item Compute $\psi'_{j,i} = \proveni{log*}^{\E,L}((\sid, i),R_j, (N_i, G_i, \Gamma_i, G); (\gamma_i, v_i))$.
\item         Send $(\Gamma_i, D_{j,i}, F_{j,i}, \hat{D}_{j,i}, \hat{F}_{j,i}, \psi_{j,i}, \hat{\psi}_{j,i}, \psi'_{j,i})$
        to $P_j$.
        \end{itemize}
        \end{itemize}

    \item[\textbf{Round 3.}] \ 
    \begin{enumerate}
        \item Upon receiving $(\Gamma_j, D_{i,j}, F_{i,j}, \hat{D}_{i,j}, \hat{F}_{i,j}, \psi_{i,j}, \hat{\psi}_{i,j}, \psi'_{i,j})$ from $P_j$, do:

        \begin{itemize}
            \item Assert $\verifyni{aff\mbox{-}g}^{\E,L}((\sid, j), R_i, (N_i, N_j, K_i, D_{i,j}, F_{i,j}, \Gamma_j), \psi_{i,j})$.
            \item Assert $\verifyni{aff\mbox{-}g}^{\E,L}((\sid, j), R_i, (N_i, N_j, K_i, \hat{D}_{i,j}, \hat{F}_{i,j}, X_j), \hat{\psi}_{i,j})$.
            \item Assert $\verifyni{log*}^{\E,L}((\sid, j), R_i, (N_j, G_j, \Gamma_j, G), \psi'_{i,j})$.
        \end{itemize}

        \item Compute $\Gamma = \sum_{j\in [n]} \Gamma_j$ and $\Delta_i = k_i \cdot \Gamma$.

        \item For $j \ne i$, do:
        \begin{itemize}
            \item Compute $\alpha_{i,j} = \dec_{(p_i,q_i)}(D_{i,j})$ and $\hat{\alpha}_{i,j} = \dec_{(p_i,q_i)}(\hat{D}_{i,j})$.
            \item Compute $\psi''_{j,i} = \proveni{log*}^{\E,L}((\sid, i), R_j,(N_i, K_i, \Delta_i, \Gamma); (k_i, \rho_i))$.
        \end{itemize}
        \item  Compute $\delta_i = \gamma_i k_i + \sum_{j \ne i}(\alpha_{i,j} + \beta_{i,j}) \bmod q$ and
            $\chi_i = x_i k_i + \sum_{j \ne i}(\hat{\alpha}_{i,j} + \hat{\beta}_{i,j}) \bmod q$. 

        \item Send $(\delta_i, \Delta_i, \psi''_{j,i})$ to each $P_j$.
    \end{enumerate}

    \item[\textbf{Output}] \
    \begin{enumerate}
        \item Upon receiving $(\delta_j, \Delta_j, \psi''_{i,j})$ from $P_j$, assert $\verifyni{log*}^{\E,L}((\sid, j), R_i,(N_j, K_j, \Delta_j, \Gamma), \psi''_{i,j})$.

        \item Compute $\delta = \sum_j \delta_j$, and do:
        \begin{itemize}
            \item Assert $\delta \cdot G = \sum_j \Delta_j$. 
            \item Set $R = \delta^{-1} \cdot \Gamma$ and output $(R, k_i, \chi_i)$.
        \end{itemize}
    \end{enumerate}
\end{description}

\subsubsection{Threshold ($t$-out-of-$n$) Presinging}

\begin{description}

\item[\bf Input.] 
  Size of signing set $t$,
  identities of parties in signing set,
  index $i\in [t]$,
  index map\footnote{$S(i)$ is the index that $P_i$ had at key-generation time.} $S : [t] \to [n]$,
  key share $K_{S(i)}$,
  context separation string $\sid$,
  security level $L$, curve $\E$ with generator $G$.

  {\bf HD-wallets inputs:} additive $\text{shift} \in \Fq$ \footnote{Deriving additive shift is up to specific standard of HD-wallets, e.g. see \cite{bip32} or \cite{slip10} \label{deriving-shift-note}} ($\text{shift} = 0$ disables HD derivation)

    \item[\bf Setup.] The key share $K_{S(i)}$ contains $\threshold$, number of key holders~$n$, secret share $x'_{S(i)}$, parties' public shares $\vec X' = (X'_j)_{j \in [n]}$, a map $I : [n] \to \Fq \setminus \{0\}$, Paillier secret key $\sk_{S(i)}$, parties' Paillier keys $\vec N' = (N'_j)_{j \in [n]}$, and parties' auxiliary information $\vec R' = (s_j, t_j, \hat N)_{j \in [n]}$.

    \item[\bf Step 1.] Set $\sk_i = \sk_{S(i)}$ and 
                       $\vec R = (R'_{S(j)})_{j \in [n]}$. Then:
    \begin{itemize}
        \item If shares are additive\footnote{In this case we have $t=n$.} shares of the private key, 
            set $x_{i} = x'_{S(i)}$, $\vec X = (X'_{S(j)})_{j \in [n]}$.
        
        \item If shares are Shamir secret shares of the private key: 
        \begin{itemize}
            \item For $j \in [t]$, compute Lagrange coefficient $\lambda_j = \prod\limits_{m \in [t]\setminus\{j\}} \frac{I(S(m))}{I(S(m)) - I(S(j))} \bmod{q}$.
            \item Compute $x_{i} = \lambda_i \cdot x'_{S(i)}$.
            \item For $j \in [n]$, compute $X_j= \lambda_j \cdot X'_{S(j)}$; then set $\vec X = \{X_j\}_{j \in [t]}$.
        \end{itemize}
    \end{itemize}

    \item[\bf Step 2.] {\bf (HD-wallets)} If HD-wallets support enabled:
    \begin{itemize}
      \item Set $X_0 := X_0 + \text{shift} \cdot G$
      \item If $i = 0$, set $x_0 := x_0 + \text{shift}$
      \item Note: output signature will be valid for public key $Y + \text{shift} \cdot G$
    \end{itemize}

    \item[\bf Step 3.] Call the non-threshold presigning protocol from the previous section using inputs $t, i, x_i, \vec X$, $\sk_i$, $\vec R, \sid, L, \E$.
\end{description}


\subsection{Signing}
\label{sec:signing}
The signing protocol has two parts: one that takes the output from the presignature protocol and a hashed message and produces a partial signature, and another that takes partial signatures and combines them to produce a signature. 
%This separation is introduced so final signature reconstruction can be done on coordinator. Functions are defined below.
    
\medskip\noindent{\bf Local signing.} The input is a presignature 
  $S_i = (R, k_i, \chi_i)$,
  a hashed message~$m$,
  and, if HD wallets support is enabled, additive $\text{shift} \in \Fq$ \footref{deriving-shift-note} ($\text{shift} = 0$ disables HD derivation).
  Do:
        \begin{itemize}
            \item If {\bf HD-wallets} support enabled: \\
              Set $\chi := \chi + k \cdot \text{shift}$
            \item Set $r = \affineX{R}$ and $\sigma_i = k m + r \chi$.
            \item Output $(r, \sigma_i)$.
        \end{itemize}
        

    \medskip\noindent{\bf Combining presignatures.} The input is the public key~$Y$, partial signatures $\{(r_i, \sigma_i)\}_{i=0}^{n-1}$, and hashed message~$m$.
The function does:
        \begin{itemize}
            \item Assert $r_0 = r_1 = \dots = r_{n-1}$.
            \item Let $\sigma = \sum_j \sigma_j$.
            \item If $(r, \sigma)$ is not a valid signature on hashed message~$m$ with respect to public key~$Y$, then abort. Otherwise, 
            output $(r, \sigma)$.
        \end{itemize}


\ignore{
\section{Further Optimizations}

\todo{decide whether we want to move it into jira tickets/github issues}

\begin{enumerate}
    \item Full use of CRT: the CRT optimization can be used (when applicable) for all computations involving exponentiation, not just Paillier encryption

    \item Combine CRT and fixed-based multiexponentiation optimizations

    \item In some cases, can do probabilistic bath checks. It may also be possible to combine several of the ZK proofs. (Note, however, that these optimizations involve deviations from~\cite{cggmp21}.)
\end{enumerate}
}

\ignore{
\appendix
\section{Original presigning protocol}
\label{appendix:presigning-orig}
Original presigning protocol defined in CGGMP paper contains many typos and mistakes. We show original protocol below. Places that were changed in our implementation are \changedt{highlighted in red}.

\begin{description}
    \item[\textbf{Input.}] Key share $K_i$, context separation string $\sid$, security level $L$, curve $E$ with generator
        $G$.
    \item[\textbf{Setup.}] $P_i$ reads key share $K_i$ containing its index $i$, secret share $x_i$, list of public 
    shares $X = (G \cdot x_j)_j$, shared public key $\pk = \sum_j X_j$, bytestring $\rid$ generated at keygen, 
    prime numbers $p_i$ and $q_i$ (Paillier secret key), list of parties public Paillier keys $N = (N_j)_j$, 
    lists of parties' Ring-Pedersen parameters $R = (N_j, s_j, t_j)_j$. 

    \item[\textbf{Round 1.}] \
    \begin{itemize}
        \item Sample $\gamma_i, k_i \gets \Fq$, $v_i, \rho_i \gets \Z^*_{N_i}$ and set $G_i = \enci{i}(\gamma_i, v_i)$,
            $K_i = \enci{i}(k_i, \rho_i)$
        \item Compute $\psi^0_{j,i} = \proveni{enc}^{L,R_j}((\sid, i), (N_i, K_i); (k_i, \rho_i))$ for every $j \ne i$
        \item Broadcast $(K_i, G_i)$ and send $\psi^0_{j,i}$ to each $P_j$
    \end{itemize}

    \item[\textbf{Round 2.}] \
    \begin{enumerate}
        \item Upon receiving $(K_j, G_j, \psi^0_{i,j})$ from $P_j$, do:
            \begin{itemize}
                \item Assert $\verifyni{enc}^{L,R_i}((\sid, j), \psi^0_{i,j}, (N_j, K_j))$ 
            \end{itemize}
        \item When passing above verification for all $P_j$, set $\Gamma_i = G \cdot \gamma_i$ and for every $j \ne i$ do:
        \begin{itemize}
            \item Sample $\changed{r_{i,j}}, s_{i,j}, \changed{\hat{r}_{i,j}}, \hat{s}_{i,j} \gets \Z_{N_j}$,
                $\beta_{i,j}, \hat{\beta}_{i,j} \gets \Z_{2^{\ell' + 1}}$
            \item $D_{j,i} = (\gamma_i \odot K_j) \oplus \enci{j}(-\beta_{i,j}, s_{i,j})$
                and $F_{j,i} = \enci{i}(\changed{\beta_{i,j}}, \note{r_{i,j}})$
            \item $\hat{D}_{j,i} = (x_i \odot K_j) \oplus \enci{j}(-\hat{\beta}_{i,j}, \hat{s}_{i,j})$
                and $\hat{F}_{j,i} = \enci{i}(\changed{\hat{\beta}_{i,j}}, \note{\hat{r}_{i,j}})$
            \item $\psi_{j,i} = \proveni{aff\mbox{-}g}^{L,R_j}((\sid, i), (N_j, N_i, \changed{D_{j,i}, K_j}, F_{j,i}, \Gamma_i); (\gamma_i, \beta_{i,j}, s_{i,j}, r_{i,j}))$
            \item $\hat{\psi}_{j,i} = \proveni{aff\mbox{-}g}^{L,R_j}((\sid, i), (N_j, N_i, \changed{\hat{D}_{j,i}, K_j}, \hat{F}_{j,i}, X_i); (x_i, \hat{\beta}_{i,j}, \hat{s}_{i,j}, \hat{r}_{i,j}))$
            \item $\psi'_{j,i} = \proveni{log*}^{L,R_j}((\sid, i), (N_i, G_i, \Gamma_i, G); (\gamma_i, v_i))$
        \end{itemize}
        Send $(\Gamma_i, D_{j,i}, F_{j,i}, \hat{D}_{j,i}, \hat{F}_{j,i}, \psi_{j,i}, \hat{\psi}_{j,i}, \psi'_{j,i})$
        to each $P_j$
    \end{enumerate}

    \item[\textbf{Round 3.}] \ 
    \begin{enumerate}
        \item Upon receiving $(\Gamma_j, D_{i,j}, F_{i,j}, \hat{D}_{i,j}, \hat{F}_{i,j}, \psi_{i,j}, \hat{\psi}_{i,j}, \psi'_{i,j})$ from $P_j$, do:

        \begin{itemize}
            \item Assert $\verifyni{aff\mbox{-}g}^{L,R_i}((\sid, j), \psi_{i,j}, (N_i, N_j, \changed{D_{i,j}, K_i}, F_{j,i}, \Gamma_j))$
            \item Assert $\verifyni{aff\mbox{-}g}^{L,R_i}((\sid, j), \hat{\psi}_{i,j}, (N_i, N_j, \changed{\hat{D}_{k,j}, K_i}, \hat{F}_{j,i}, X_j))$
            \item Assert $\verifyni{log*}^{L,R_i}((\sid, j), \psi'_{i,j}, (N_j, G_j, \Gamma_j, G))$
        \end{itemize}

        \item When passing above verification for all $P_j$, set $\Gamma = \sum_j \Gamma_j$ and $\Delta_i = k_i \cdot \Gamma$,
        and do:

        For every $j \ne i$, set $\alpha_{i,j} = \dec(D_{i,j})$ and $\hat{\alpha}_{i,j} = \dec(\hat{D}_{i,j})$ and:
        \begin{itemize}
            \item $\delta_i = \gamma_i k_i + \sum_{j \ne i}(\alpha_{i,j} - \beta_{i,j}) \bmod q$
            \item $\chi_i = x_i k_i + \sum_{j \ne i}(\hat{\alpha}_{i,j} - \hat{\beta}_{i,j}) \bmod q$
            \item $\psi''_{j,i} = \proveni{log*}^{L,R_j}((\sid, i), (N_i, K_i, \Delta_i, \Gamma); (k_i, \rho_i))$
        \end{itemize}

        Send $(\delta_i, \Delta_i, \psi''_{j,i})$ to each $P_j$
    \end{enumerate}

    \item[\textbf{Output}] \
    \begin{enumerate}
        \item Upon receiving $(\delta_j, \Delta_j, \psi''_{i,j})$ from $P_j$, do:
        \begin{itemize}
            \item Assert $\verifyni{log*}^{L,R_i}((\sid, j), \psi''_{i,j}, (N_j, K_j, \Delta_j, \Gamma))$
        \end{itemize}

        \item When passing above verification for all $P_j$, set $\delta = \sum_j \delta_j$, and do:
        \begin{itemize}
            \item Assert $\delta \cdot G = \sum_j \Delta_j$. In case of failure, return error that protocol was aborted by unknown party.
            \item Set $R = \delta^{-1} \cdot \Gamma$ and output $(R, k_i, \chi_i)$
        \end{itemize}
    \end{enumerate}
\end{description}
}

\bibliographystyle{plain}
\bibliography{ref}
\end{document}

\begin{appendix}
    \section{(Non-Threshold) Key Refresh}
\label{sec:refresh}

During a key refresh, shares of a private key are refreshed (while still maintaining a sharing of the same key). %Note that secure deletion of the old key shares is handled by the caller, and is out of scope of our library.
When key refresh is used it is likely because of a concern that some signer has been compromised; thus, as noted earlier, 
one should probably do a key refresh for \emph{all} the keys stored by that cluster, and should also re-run the provisioning protocol for that cluster.


The following is based on \cite[Figure~6]{cggmp21}, with some unnecessary steps eliminated.

%Compared to the version in the CGGMP21 paper, there are changes in round 3 related to proof recipients.
%The original paper says that $\Pi_{sch}$ proof for $x_i^j$ should be calculated and sent to each recipient such that a recepient only receives one proof from each party. However, in the next round the original paper assumes that the recipient received a vector of proofs from each party. Here we take that assumption as the correct intent and make each party broadcast all proofs.

\begin{description}
  \item[\textbf{Input.}]
    Key share $K_i$ (containing
      a party index $i$,
      secret share~$x_i$,
      and list $\vec{X}=(X_k)_{k \in [n]}$ of public shares),
    context separation string $\sid$,
    security level~$L$,
    curve $E$ with generator $G$ of prime order~$q$.


  \item[\textbf{Round 1.}] \
    \begin{itemize}
%     \item Let $p_i, q_i$ be $4\kappa$-bit safe primes that were pre-generated. 
  %    \item Compute $N_i = p_iq_i$ and $\phi = (p_i - 1)(q_i - 1)$, and create a Paillier decryption key using those values.
      \item Sample $x_i^1, \ldots, x_i^{n-1} \leftarrow \Z_q$ and set $x_i^0=-\sum_{k=1}^{n-1} x_i^k$.
        Then compute $X_i^k = x_i^k \cdot G$ for $k \in [n]$ and set $\vec{X}_i = (X_i^k)_{k \in [n]}$. 

  %    \item Sample  $r \gets \Z_{N_i}^{*}$ and $\lambda \gets \Z_\phi$, and compute $t_i = r_i^2 \bmod N_i$ and $s_i = t_i^\lambda \bmod N_i$.
   %   \item Compute $\hat{\psi}_i = \proveni{prm}^L(\sid||i, (N_i, s_i, t_i), (\phi, \lambda))$.
      \item For $j \in [n]\setminus \{i\}$, 
      compute $(A_i^j, \tau_i^j) \gets \commit{sch}()$. Set
        $\vec{A_i} = (A_i^j)_{j \in [n]}, \vec{\tau} = (\tau_i^j)_{j \in [n]}$.
        %\jnote{We could potentially optimize this so only one proof is needed}
      \item Sample $\rho_i \gets \{0,1\}^\kappa$.
      \item Sample nonce $u_i \gets \{0,1\}^\kappa$ % another value in code
        and compute $V_i =
        H(\sid\| n\| i\| \vec{X}_i\| \vec{A}_i\| N_i\| s_i\| t_i\| \hat{\psi}_i\| \rho_i\| u_i)$.
      \item Broadcast $V_i$ to all parties.
    \end{itemize}

    \item[\bf Reliability check.] Optionally, if reliability check is enabled:
    \begin{itemize}
        \item Upon receiving $V_j$ from all parties:
        set $h_i = H(\Encode{echo}(V_0, \dots, V_{n-1}))$
            and send $h_i$ to all parties.
        
        \item Upon receiving $h_j$ from all  parties:
        abort if $h_i \neq h_j$ for some $j \in [n]$.
    \end{itemize}

  \item[\textbf{Round 2.}] \

    \begin{itemize}
      \item Receive $V_j$ from all parties.
      \item Send $(\vec{X}_i, \vec{A}_i, N_i, s_i, t_i, \hat{\psi}_i, \rho_i, u_i)$ to all parties.
    \end{itemize}

  \item[\textbf{Round 3.}] \

    \begin{itemize}
      \item Receive $(\vec{X}_j, \vec{A}_j, N_j, s_j, t_j, \hat{\psi}_j, \rho_j, u_j)$
        from all parties.
      \item For all $j \in [n]$, set $R_j = (N_j, s_j, t_j)$; let $\vec{R} = (R_j)_{j \in [n]}$.

      \item For each $j \in [n]$:
        \begin{itemize}
          \item Assert $V_j =
            H(\sid\| n\| j\| \vec{X}_j\| \vec{A}_j\| N_j\| s_j\| t_j\| \hat{\psi}_j\| \rho_j\| u_j)$.
          \item Assert that $\vec X_j$ and $\vec A_j$ each contain exactly $n$ elements.
          \item Assert that $N_j$ is at least $8 \cdot \kappa - 1$ bits in length.
          \item Assert $\verifyni{prm}^L((\sid, j), (N_j, s_j, t_j), \hat{\psi}_j)$.
          \item Assert $\sum_k X_j^k = 0$ (the identity in~$\E$).
          \item Construct Paillier encryption key from $N_j$.
        \end{itemize}

      
      \item Compute $\rho = \oplus_j \rho_j$.
      \item Compute $\psi_i = \proveni{mod}^L((\sid, i, \rho), N_i, (p_i, q_i))$.
      \item For $j \neq i$ do:
        \begin{itemize}
        \item Compute $\phi_i^j = \proveni{fac}^{L}((\sid, i, \rho), R_j, N_i, (p_i, q_i))$. 
        \item Compute $C_i^j = \enc_j(x_i^j)$ using a random nonce.
        \item Compute $e^j_i = H(\sid || i || \rho\|X_i^j\|A_i^j)$ and $\psi_i^j = \prove{sch}(\tau_i^j, e_i^j, x_i^j)$.

          \item Send $(\psi_i, \phi_i^j, C_i^j, \{\psi_i^k\}_{k \in [n]})$ to $P_j$.
        \end{itemize}

    \end{itemize}

  \item[\textbf{Output.}] \

    \begin{itemize}
      \item Receive $(\psi_j, \phi_j^i, C_j^i, \{\psi^k_j\}_{k \in [n]})$ from all parties.
    
      \item For $j \in [n]$ do:
        \begin{itemize}
        \item  Decrypt $C_j^i$ to obtain $x_j^i$ and
          assert
            $X_j^i = x_j^i \cdot G$.
                     \item Assert $\verifyni{mod}^L((\sid, j, \rho), N_j, \psi_j)$.
          \item Assert $\verifyni{fac}^{L}((\sid, j, \rho), R_i, N_j, \phi_j^i)$.
          \item For $k \in [n]$, set $e_j^k = H(\sid || j || \rho\|X_j^k\|A_j^k)$ and 
          assert $\verify{sch}(\psi_j^k, A_j^k, e^k_j, X_j^k)$.
        \end{itemize}

      \item Compute $x_i^* = x_i + \sum_{j\in [n]} x_j^i$.
      \item Compute $\vec{X}^* = (X_k^*)_{k \in [n]}$, where
        $X_k^* = X_k + \sum_{j\in [n]} X_j^k$.
      \item Return 
        $((i, x_i^*, \vec{X}^*), p_i, q_i, \vec{R})$.
    \end{itemize}

\end{description}

\end{appendix}

\end{document}
